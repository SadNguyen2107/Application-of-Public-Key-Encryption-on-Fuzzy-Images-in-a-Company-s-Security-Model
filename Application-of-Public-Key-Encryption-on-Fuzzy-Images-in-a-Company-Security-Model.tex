\documentclass[graybox]{svmult}

\begin{document}

\title*{Application of Public Key Encryption on Fuzzy Images in a
    Company's Security Model}
\author{Name of Author}
\institute{Name of Institute}

\maketitle

\abstract{
    This report presents an approach to addressing the implementation of a
    security model in companies by using picture fuzzy public key encryption, with a more
    detailed focus on utilizing user biometric data as a secret key. Since biometric data is
    blurred or noisy and changes with each collection, traditional public key encryption
    models cannot be used; instead, a picture fuzzy public key encryption model must be
    employed. This study introduces the concept of picture fuzzy public key encryption
    (PFPKE), a public encryption model that accepts a portion of blurred data (a noisy version
    of the original biometric data) as a private key for decrypting the ciphertext. Unlike
    traditional public key encryption models, where the private key is typically stored on
    devices (e.g., on USB drives), the picture fuzzy public key encryption model does not
    require any device to store the private key
}

\keywords{Picture fuzzy public key encryption, fuzzy data, biometrics}

\section{Introduction}
In traditional security models within companies using public key infrastructure, each employee must have a public-private key pair. If an employee receives an encrypted message, it means the message has been encrypted using that employee's public key, and they will decrypt it with their private key. The most important aspect for the employee is to keep their private key secure, as a leak of this key would compromise the system's security. A widely accepted method is to store the private key on a physical device like a smart card or USB drive and require the employee to remember a password to activate it \cite{Ellison2000}.

An ideal approach is to use biometric data (e.g., fingerprints or iris patterns) \cite{Connaughton2007} as a private key since biometric data is unique to each individual, providing a convenient and secure way to serve as a private key for users. However, biometric data can be blurry or noisy and may change every time it is captured, making it unsuitable for use as a private key in traditional public key encryption schemes.

To address this issue, this paper introduces the concept of fuzzy signatures \cite{Takahashi2015}, which use biometric data as a private key without requiring any assistance \cite{Dodis2008}. Thus, it applies public key cryptography with fuzzy images \cite{Son2016} in internal company security models utilizing biometrics.

\section{Symbols and definitions}
Part 2:

\section{Famework and security model of public key cryptography of watermark}
Part 3:

\section{Construct algorithm for fuzzy public key encryption}
Part 4:

\section{Some additional properties}
Part 5:

\section{Conclusion}
In the traditional method, messages encrypted using public key schemes rely on protecting the privacy of the user's private key by storing it in a physical device, such as a USB token carried by the user. However, it is not always practical for the user to keep the device with them at all times. To solve this problem, using individual biometric data as the private key is a reasonable alternative. 

However, biometric data can change every time it is collected, making it unsuitable for direct use as a private key. In this paper, the concept of public key cryptography with fuzzy data is introduced, where a part of the biometric data can be used as the private key to decrypt ciphertexts without requiring any additional information. 

Compared to traditional public key encryption, the primary advantage of fuzzy public key encryption is that it does not require the user to carry any memory device or password to function as a private key. When using fuzzy public key encryption, attention should be paid to the value of the fuzzy set in the neutral degree, as these unclear points in system access allow potential vulnerabilities where hackers could access the system.


\section*{Acknowledgments}
Acknowledgments section


\begin{thebibliography}{99}
    \bibitem{Ellison2000} Ellison, C., Schneier, B. (2000). Ten risks of PKI: What you're not being told about public key infrastructure. \textit{Computer Security Journal}, 16.

    \bibitem{Connaughton2007} Connaughton, R., Bowyer, K.W., Flynn, P.J. (2007). Fusion of face and iris biometrics. In: \textit{Handbook of Iris Recognition}.

    \bibitem{Dodis2008} Dodis, Y., Ostrovsky, R., Reyzin, L., Smith, A.D. (2008). Fuzzy extractors: How to generate strong keys from biometrics and other noisy data. \textit{SIAM Journal on Computing}.

    \bibitem{Takahashi2015} Takahashi, K., Matsuda, T., Murakami, T., Hanaoka, G., Nishigaki, M. (2015). A signature scheme with a fuzzy private key.

    \bibitem{Son2016} Son, L.H., Viet, P.V., Hai, P.V. (2016). Picture inference system: A new fuzzy inference system on picture fuzzy set. \textit{Applied Intelligence}.

    \bibitem{Sahai2005} Sahai, A., Waters, B. (2005). Fuzzy identity-based encryption. In: \textit{Advances in Cryptology - EUROCRYPT 2005}, 24th Annual International Conference on the Theory and Applications of Cryptographic Techniques, Aarhus, Denmark.

    \bibitem{MacKenzie2004} MacKenzie, P.D., Reiter, M.K., Yang, K. (2004). Alternatives to non-malleability: Definitions, constructions, and applications (extended abstract).

    \bibitem{Gamal1985} ElGamal, T. (1985). A public key cryptosystem and a signature scheme based on discrete logarithms.

    \bibitem{Cuong2014} Cuong, B.C. (2014). Picture fuzzy sets. \textit{Journal of Computer Science and Cybernetics}.

    \bibitem{Matsuda2016} Matsuda, T., Takahashi, K., Murakami, T., Hanaoka, G. (2016). Fuzzy signatures: Relaxing requirements and a new construction.
\end{thebibliography}


\end{document}